%این فایل از روی آموزش‌های لتکس در کانال آپارات دکتر مس‌فروش (عضو هیات علمی دانشگاه صنعتی شاهرود) ساخته شده است

%برای ساخت فصل‌ها باید از کلاس کتاب استفاده کنیم
\documentclass[12pt]{book}
%اضافه شدن زیربخش‌های درون متن تا ۴ لایه
\setcounter{secnumdepth}{4}
%اضافه شدن زیربخش‌ها تا ۴ لایه برای فهرست مطالب
\setcounter{tocdepth}{4}

%واردکردن بسته های مختلف، بسته xepersian را بعد از همه بسته‌ها وارد کنید

%برای قابل کلیک شدن ارجاعات و برچسب‌ها از پیکیج زیر استفاده می‌کنیم
\usepackage{hyperref}
\usepackage{xepersian}


\settextfont[Path=./fonts/,
BoldFont={XB Yas Bd}, 
ItalicFont={XB Yas It},
BoldItalicFont={XB Yas BdIt},
Extension = .ttf]{XB Yas}
%تعریف فونت جدید
%اگر فونت‌های زیر را نصب کردید می‌توانید خط‌های زیر را از حالت کامنت خارج کنید
%\defpersianfont\nast{IranNastaliq}
%\defpersianfont\titr{B Titr Bold}


\title{پایان‌نامه}
\author{مهدی اصلانی}
\date{۲۵ تیر ۱۴۰۰}

\begin{document}
\maketitle

%برای ساخت فصل از دستور chapter استفاده می‌کنیم
\chapter{دستورات پایه در \LaTeX}
%برای ساخت بخش از دستور section استفاده می‌کنیم
\section{اندازه ها مختلف، شکستن خط، رفتن به پاراگراف یا خط بعد }


متن با اندازه نرمال
{\LARGE متن با اندازه بزرگ}
{\tiny  متن با اندازه کوچولو}
 {\footnotesize متن با اندازه پاورقی}
سایر اندازه ها را می‌توانید در اینترنت پیدا کنید (
\lr{footbote, large, LARGE, huge, HUGE, ...}
)
  برای وارد کردن تاریخ روز می توان از این دستور استفاده کرد
( \today )
و با خالی گذاشتن یک خط می‌توان به پاراگراف بعدی رفت. برای رفتن به خط بعدی می‌توانیم از دو بَک اِسلش \\ استفاده کنیم.


\section{متن های ضخیم، خوابیده، زیرخط‌دار، درونِ کادر و تاکید‌شده}\label{boldItalicUnderline}
می‌توانید به این صورت  یک نوشته را 
\textbf{بولد (ضخیم)}
یا
\textit{ایتالیک (کج)}
یا
\underline{زیرخط‌دار}
یا
\fbox{درون کادر}
یا
\emph{تاکید شده}
بنویسید.


\section{استفاده از فونت تعریف شده در قسمت preamble}
می‌توان از فونت تعریف شده در قسمت preamble استفاده کرد(قبل از استفاده از فونت‌ها مطمئن شوید آنها را نصب کرده‌اید):\\
%اگر فونت B Titr Bold و IranNastaliq را نصب کرده‌اید می‌توانید آنرا از حالت کامنت خارج کنید (در قسمت preamble هم همینکار را انجام دهید)
%{\nast ساقیا مرد نکونام نمیرد هرگز}\\
%{\titr این یک تیتر با فونت تیتر ضخیم است}

\newpage

\section{وسط‌چین - راست چین - چپ چین}
%برای ساخت زیر بخش از دستور subsection استفاده می‌کنیم
\subsection{وسط‌‌چین}\label{centeringSubsection}
دو روش مرسوم برای وسط‌چین کردن:\\
روش اول:\\
\begin{center}
	این یک متن وسط‌چین است که با روش اول ساخته شده‌ است
\end{center}
روش دوم:\\
\centerline{	این یک متن وسط‌چین است که با روش دوم ساخته شده‌ است}
%روش سوم:\\
%\centering این یک متن وسط‌چین است که با روش دوم ساخته شده‌ است
\subsection{راست‌چین و چپ‌چین}
از روش‌های بخش
\ref{centeringSubsection}
می‌توان برای چپ‌چین یا راست‌چین کردن متن نیز استفاده کرد:\\
\begin{flushleft}
		این یک متن چپ‌چین است که با روش اول ساخته شده‌ است
\end{flushleft}

یک متن معمولی
\begin{flushright}
			این یک متن راست‌چین است که با روش اول ساخته شده‌ است
\end{flushright}
%بهمین صورت می‌توان زیرزیربخش نیز تولید کرد
\subsubsection{زیرزیربخش جدید}
اگر زیرزیربخشی ساخته نشد در قسمت 
preamble
باید دو دستور زیر را وارد کنید:
1:
%\setcounter{secnumdepth}{4}
%\setcounter{tocdepth}{4}
\begin{center}
	\lr{$\backslash$setcounter \{ secnumdepth\} \{4\} } \\
	\lr{$\backslash$setcounter \{ tocdepth\} \{4\} }
\end{center}
%ارجاع دادن به بخش یا section بدون اینکه بدانیم شماره بخش چند است
در بخش
\ref{boldItalicUnderline}
دیدیم که چگونه می‌توانیم کلمات را بصورت ضخیم یا خوابیده بنویسیم.
%ارجاع دادن به فصل فرمول نویسی بدون اینکه بدانیم شماره فصل چند است
در فصل
\ref{writingFormula}
به نحوه فرمول نویسی در
\LaTeX
خواهیم پرداخت.

\section{پانویس یا پاورقی}
می‌توان برای معادله شرودینگر
\LTRfootnote{Schrödinger equation}
یک پانویس انگلیسی نوشت.
یا برای قانون گاوس
\footnote{برای مطالعه بیشتر به کتاب الکترودینامیک جکسون مراجعه کنید}
یک پاورقی فارسی نوشت.





\chapter{فرمول نویسی در \LaTeX}\label{writingFormula}




\end{document}