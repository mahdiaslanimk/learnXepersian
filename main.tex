\documentclass[12pt]{article}
\usepackage{xepersian}
\settextfont[Path=./fonts/,
BoldFont={XB Yas Bd}, 
ItalicFont={XB Yas It},
BoldItalicFont={XB Yas BdIt},
Extension = .ttf]{XB Yas}
%تعریف فونت جدید
\defpersianfont\nast{IranNastaliq}


\title{پایان‌نامه}
\author{مهدی اصلانی}
\date{۲۵ تیر ۱۴۰۰}

\begin{document}
\maketitle

\section{اندازه ها مختلف، شکستن خط، رفتن به پاراگراف یا خط بعد }


متن با اندازه نرمال
{\LARGE متن با اندازه بزرگ}
{\tiny  متن با اندازه کوچولو}
 {\footnotesize متن با اندازه پاورقی}
سایر اندازه ها را می‌توانید در اینترنت پیدا کنید (
\lr{footbote, large, LARGE, huge, HUGE, ...}
)
  برای وارد کردن تاریخ روز می توان از این دستور استفاده کرد
( \today )
و با خالی گذاشتن یک خط می‌توان به پاراگراف بعدی رفت. برای رفتن به خط بعدی می‌توانیم از دو بَک اِسلش \\ استفاده کنیم.


\section{متن های ضخیم، کج (خوابیده)، زیرخط‌دار، درونِ کادر و تاکید‌شده و سلام}
می‌توانید به این صورت  یک نوشته را 
\textbf{بولد (ضخیم)}
یا
\textit{ایتالیک (کج)}
یا
\underline{زیرخط‌دار}
یا
\fbox{درون کادر}
یا
\emph{تاکید شده}
بنویسید.


\section{استفاده از فونت تعریف شده در قسمت preamble}
می‌توان از فونت تعریف شده در قسمت preamble استفاده کرد
{\LARGE \nast ساقیا مرد نکونام نمرید هرگز}

\end{document}